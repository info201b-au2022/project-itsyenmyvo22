% Options for packages loaded elsewhere
\PassOptionsToPackage{unicode}{hyperref}
\PassOptionsToPackage{hyphens}{url}
%
\documentclass[
]{article}
\usepackage{amsmath,amssymb}
\usepackage{lmodern}
\usepackage{iftex}
\ifPDFTeX
  \usepackage[T1]{fontenc}
  \usepackage[utf8]{inputenc}
  \usepackage{textcomp} % provide euro and other symbols
\else % if luatex or xetex
  \usepackage{unicode-math}
  \defaultfontfeatures{Scale=MatchLowercase}
  \defaultfontfeatures[\rmfamily]{Ligatures=TeX,Scale=1}
\fi
% Use upquote if available, for straight quotes in verbatim environments
\IfFileExists{upquote.sty}{\usepackage{upquote}}{}
\IfFileExists{microtype.sty}{% use microtype if available
  \usepackage[]{microtype}
  \UseMicrotypeSet[protrusion]{basicmath} % disable protrusion for tt fonts
}{}
\makeatletter
\@ifundefined{KOMAClassName}{% if non-KOMA class
  \IfFileExists{parskip.sty}{%
    \usepackage{parskip}
  }{% else
    \setlength{\parindent}{0pt}
    \setlength{\parskip}{6pt plus 2pt minus 1pt}}
}{% if KOMA class
  \KOMAoptions{parskip=half}}
\makeatother
\usepackage{xcolor}
\usepackage[margin=1in]{geometry}
\usepackage{graphicx}
\makeatletter
\def\maxwidth{\ifdim\Gin@nat@width>\linewidth\linewidth\else\Gin@nat@width\fi}
\def\maxheight{\ifdim\Gin@nat@height>\textheight\textheight\else\Gin@nat@height\fi}
\makeatother
% Scale images if necessary, so that they will not overflow the page
% margins by default, and it is still possible to overwrite the defaults
% using explicit options in \includegraphics[width, height, ...]{}
\setkeys{Gin}{width=\maxwidth,height=\maxheight,keepaspectratio}
% Set default figure placement to htbp
\makeatletter
\def\fps@figure{htbp}
\makeatother
\setlength{\emergencystretch}{3em} % prevent overfull lines
\providecommand{\tightlist}{%
  \setlength{\itemsep}{0pt}\setlength{\parskip}{0pt}}
\setcounter{secnumdepth}{-\maxdimen} % remove section numbering
\ifLuaTeX
  \usepackage{selnolig}  % disable illegal ligatures
\fi
\IfFileExists{bookmark.sty}{\usepackage{bookmark}}{\usepackage{hyperref}}
\IfFileExists{xurl.sty}{\usepackage{xurl}}{} % add URL line breaks if available
\urlstyle{same} % disable monospaced font for URLs
\hypersetup{
  pdftitle={index.Rmd},
  pdfauthor={Xiaotong(Katherine) Zhang, Yenmy Vo, Natalie Hinds, Kasper Li},
  hidelinks,
  pdfcreator={LaTeX via pandoc}}

\title{index.Rmd}
\author{Xiaotong(Katherine) Zhang, Yenmy Vo, Natalie Hinds, Kasper Li}
\date{November 14, 2022}

\begin{document}
\maketitle

\hypertarget{gun-violence-in-the-us-project-proposal}{%
\section{\texorpdfstring{\textbf{Gun Violence in the US: Project
Proposal}}{Gun Violence in the US: Project Proposal}}\label{gun-violence-in-the-us-project-proposal}}

\hypertarget{code-name-waterguns}{%
\subsection{Code Name: waterguns}\label{code-name-waterguns}}

\hypertarget{authors}{%
\subsection{\texorpdfstring{\emph{Authors:}}{Authors:}}\label{authors}}

\textbf{Natalie Hinds
(\href{mailto:Nhinds2@uw.edu}{\nolinkurl{Nhinds2@uw.edu}}) Kasper
Li(\href{mailto:jiaxul9@uw.edu}{\nolinkurl{jiaxul9@uw.edu}}) Katherine
Zhang (\href{mailto:Xiaotz7@uw.edu}{\nolinkurl{Xiaotz7@uw.edu}}) Yenmy
Vo (\href{mailto:yenmyvo@uw.edu}{\nolinkurl{yenmyvo@uw.edu}})}

\hypertarget{info-201-technical-foundations-of-informatics---the-information-school---university-of-washington}{%
\subsubsection{INFO-201: Technical Foundations of Informatics - The
Information School - University of
Washington}\label{info-201-technical-foundations-of-informatics---the-information-school---university-of-washington}}

November 2022

\hypertarget{summary-information}{%
\subsection{Summary Information}\label{summary-information}}

We use datasets that are specific to incidents of gun violence, such as
the state where the violence occurred, the date it occurred, how many
casualties there were, how many areas sold guns to 18+, how many areas
sold guns to 21+, and local poverty rates. The dataset we chose includes
U.S. gun violence incidents from 2013 to 2018. To demonstrate the main
purpose of what our project is studying, we chose to calculate five
variables through the dataset that are representative of our research
questions. These are the states in the dataset with the highest number
of deaths due to gun violence, the states with the highest poverty
rates, the highest number of deaths in a day, the number of deaths in
areas where guns are sold to people over 18 years old, and the number of
deaths in areas where guns are sold to people over 21 years old. The
results obtained were \textbf{115710} deaths in areas where guns were
sold to people over the age of 18 and \textbf{289} deaths in areas where
guns were sold to people over the age of 21. The data set showed
\textbf{50} gun violence deaths in a day, and the state with the highest
number of deaths due to gun violence was \textbf{Florida}. In addition,
the state with the highest poverty rate was \textbf{New Hampshire}.

\hypertarget{summary-table}{%
\subsubsection{Summary Table}\label{summary-table}}

\begin{verbatim}
##   deaths_sales_18 deaths_sales_21   state       state.1 n_killed
## 1          115710             289 Florida New Hampshire       50
\end{verbatim}

\begin{itemize}
\tightlist
\item
  This table shows the five values we chose to calculate. As you can
  visually see from the table, the number of gun violence deaths in
  areas where guns are sold to those over 18 is much higher than the
  number of deaths in areas where guns are sold to those over 21. Also,
  it can be concluded from the maximum number of deaths in one day due
  to gun violence that gun violence is a very serious cause for concern.
\end{itemize}

\hypertarget{charts}{%
\subsection{Charts}\label{charts}}

\hypertarget{chart1-poverty}{%
\subsubsection{Chart1: Poverty}\label{chart1-poverty}}

\includegraphics{index_files/figure-latex/unnamed-chunk-3-1.pdf}

\begin{itemize}
\tightlist
\item
  This chart was included to show the relationship between gun violence
  and poverty levels. The chart reveals that states with higher poverty
  levels tend to have more gun deaths. New Hampshire, North Dakota, and
  Wyoming are some of the states with the highest percentage of people
  above poverty. These are also the states with some of the lowest gun
  deaths. There's a trend that the midwest has a low number of gun
  deaths while the eastern states as well as the states lining the
  southern border have high gun deaths.
\end{itemize}

\hypertarget{chart2-universal-background-checks-and-gun-deaths.}{%
\subsubsection{Chart2: Universal Background Checks and Gun
Deaths.}\label{chart2-universal-background-checks-and-gun-deaths.}}

\includegraphics{index_files/figure-latex/unnamed-chunk-4-1.pdf}

\begin{itemize}
\tightlist
\item
  I chose to include the chart to show the amount of deaths and the
  effect that background checks have on deaths in each state. The
  information reveals that only 9 states have implemented gun control
  laws with universal background checks. The chart also reveals that
  many states without universal background checks have a significant
  proportion of deaths compared to states with universal background
  deaths, with an exception of California. The data visualization does
  not account for population density, as states with rural populations
  tend to have a lower amount of gun violence.
\end{itemize}

\hypertarget{chart3-the-number-of-gun-violence-injured}{%
\subsubsection{Chart3: The number of gun violence \&
injured}\label{chart3-the-number-of-gun-violence-injured}}

\includegraphics{index_files/figure-latex/unnamed-chunk-5-1.pdf}

\begin{itemize}
\tightlist
\item
  This chart shows the number of people killed and injured by shootings
  across the United States over the years. Since there are many
  variables in the original data, I simplified it to monthly data. The
  chart shows that the number of injuries and deaths from 2013 to 2014
  is not very large. But since 2014, the number has soared, and the
  trend has been increasing year by year. For example, in 2014, the
  maximum number of people injured in one month was about 2,500, but in
  2016, the maximum number of people injured in one month rose to 3,000.
  Therefore, we can conclude from this data visualization that gun
  violence is gradually getting worse.
\end{itemize}

\end{document}
